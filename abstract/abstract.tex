\addcontentsline{toc}{chapter}{Abstract}

\begin{abstract}
% Domain and motivation 
Forced migrants face many challenges when trying to settle into life in a new city. Social isolation is one common consequence of the upheaval they experience.
% Problem
Research has shown that technology can ease various parts of the resettlement process, but work is needed to see how it can address social isolation in particular. We hypothesized that a location-based freecycling service would be particularly suitable for this purpose, due to freecycling's potential to bolster social engagement and location-based services' ability to adapt to the user's context. 
% Solution
We conducted needs assessment interviews with five forced migrants and six freecyclers in Münster, Germany. We analyzed the results of the interviews to develop user requirements for a theoretical service. We implemented a subset of the user requirements as part of a prototype mobile app. Then we evaluated the app with 6 forced migrants and 16 freecyclers during a two-week trial.
% Results
Our investigation showed that, with careful design, a location-based freecycling service can meet the needs of both locally established freecyclers and forced migrants seeking to reduce their social isolation.
% Contributions
These findings contribute to literature on the needs of forced migrants and how to meet those needs with geospatial technologies. Our findings can benefit researchers of forced migrant social isolation and developers of location-based and other services to support forced migrant resettlement.

\end{abstract}