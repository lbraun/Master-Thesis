\chapter{Prototype Evaluation Questionnaires}
\label{app:questionnaires}

\section{Demographic Questionnaire}
\label{app:questionnaires-demographic}

\begin{enumerate}
    \item What is your age?

    (Under 18, 18 to 25, 26 to 35, 36 to 45, 46 to 55, Over 55)

    \item What is your gender?

    (Male, Female, Other)

    \item In which city or cities did you grow up?

    (Open question)

    \item Are you an asylum seeker or refugee?

    (Yes, No)

    \item Have you ever used Verschenk's Münster, Foodsharing, Freecycling.org, or any other ``freecycling'' system?

    (Yes, No)
\end{enumerate}


\section{Lubben Social Network Scale Questionnaire}
\label{app:questionnaires-lsns}

This questionnaire is the result of an effort to develop a quick measure of social isolation to help identify suicide candidates among seniors from mainland China \cite{chang_validation_2018}. For all questions, the answer possibilities are ``none,'' ``one,'' ``two,'' ``three or four,'' ``five to eight,'' or ``nine or more''.

\noindent \textit{FAMILY: Considering the people to whom you are related by birth, marriage, adoption, etc...}
\begin{enumerate}
    \item How many relatives do you see or hear from at least once a month?
    \item How many relatives do you feel at ease with that you can talk about private matters?
    \item How many relatives do you feel close to such that you could call on them for help?
\end{enumerate}

\noindent \textit{FRIENDSHIPS: Considering all of your friends including those who live in your neighborhood...}
\begin{enumerate}
    \setcounter{enumi}{3}
    \item How many of your friends do you see or hear from at least once a month?
    \item How many friends do you feel at ease with that you can talk about private matters?
    \item How many friends do you feel close to such that you could call on them for help?
\end{enumerate}



\section{Offer Review Questionnaire}
\label{app:questionnaires-review}

This questionnaire appeared to participants of the prototype evaluation study as soon as they tapped a button in the app to say that they had given their offer to another user.

\begin{enumerate}
    \item To whom did you give your offer?

    (Select user from a dropdown)

    \item How did you get in contact?

    (Phone or SMS, WhatsApp, Email, Facebook)

    \item Where did you meet up?

    (At my home, At their home, At someone else's home, At my work, At their work, In another public place, We didn't meet in person)

    \item How satisfied were you with the interaction?

    (Very satisfied, Satisfied, Slightly dissatisfied, Very dissatisfied)

    \item How likely are you to contact this person again?

    (Very likely, Likely, Unlikely, Very unlikely)
\end{enumerate}




\section{Usability Questionnaire}
\label{app:questionnaires-usability}

We used the System Usability Scale (SUS) to measure the usability of our prototype. This scale and the associated questions were originally developed by \cite{brooke_sus_1996} and has become the industry standard for measuring usability \cite{bangor_empirical_2008}. Participants were asked to answer all questions on a scale from one to five, where one meant ``strongly agree'' and five meant ``strongly disagree''.

\begin{enumerate}
    \item I think that I would like to use this system frequently.
    \item I found the system unnecessarily complex.
    \item I thought the system was easy to use.
    \item I think that I would need the support of a technical person to be able to use this system.
    \item I found the various functions in this system were well integrated.
    \item I thought there was too much inconsistency in this system.
    \item I would imagine that most people would learn to use this system very quickly.
    \item I found the system very cumbersome to use.
    \item I felt very confident using the system.
    \item I needed to learn a lot of things before I could get going with this system.
\end{enumerate}




\section{Usefulness Questionnaire}
\label{app:questionnaires-usefulness}

We developed the following nine questions in attempt to quickly assess changes in the indicators of social isolation established by \citeA{cornwell_measuring_2009}. As with the usability questionnaire, participants were asked to answer all questions on a scale from one to five, where one meant ``strongly agree'' and five meant ``strongly disagree''.

\begin{enumerate}
    \item I think this app increased the size of my social network in Münster.
    \item I made contact with people outside of my normal circles through this system.
    \item I think using this system increased my contact with others during the last two weeks.
    \item I feel like I have a lot in common with other people using this system.
    \item I think the contacts made through this app are likely to lead to new friendships.
    \item I feel like part of a community while using this system.
    \item I think I can trust the other users of this system.
    \item I found I could rely on the other users of this system.
    \item I think this app made me feel less isolated from others in Münster.
\end{enumerate}
