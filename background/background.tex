\chapter{Related Work}
\label{cha:background}

The design of a location-based freecycling service that reduces the social isolation of forced migrants requires 
a) an understanding of social isolation as it occurs in forced migrant communities, 
b) a consideration of the best practices and common concerns when developing ICT to address forced migrant challenges,
c) an evaluation of the unique strengths and weaknesses of geospatial technologies in this area,
d) an understanding of the strengths and limitations of freecycling platforms for promoting positive social contact, and
e) the selection of a design process that effectively meets forced migrant needs.
Following is a short summary of past research on each of these five topics.

\section{Social Isolation and Forced Migrants}
\label{sec:social_isolation}


\subsection*{Introduction}

Social isolation is defined as ``the absence of social interactions, contacts, and relationships with family and friends, with neighbors on an individual level, and with `society at large' on a broader level'' \cite{berg_social_1992}.

According to \citeA{beacker_technology_2014}, social isolation is not the same thing as loneliness. One can feel lonely even when one has frequent contact with others. One can also choose to isolate oneself from others and not feel lonely. However, when one is forced into social isolation, as is often the case with forced migrants, the isolation commonly leads to loneliness.

\citeA{cornwell_measuring_2009} consider loneliness to be a part of social isolation, but still divide social isolation into two components: social disconnectedness and perceived isolation. Social disconnectedness implies either a small social network or a lack of activity within one's social network. Perceived isolation comes from a lack of support and loneliness.

Similarly, forced migrant integration can be divided into two categories, practical integration and emotional integration \cite{colic-peisker_croatians_2002}. Social disconnectedness is an obstacle to practical integration and loneliness is an obstacle to emotional integration. Clearly, both are important and interrelated. In this thesis, however, we focus primarily on social disconnectedness because it is simpler to measure (e.g. by counting instances of social contact) and because changes are more readily observed, even over a short period of time.


\subsection*{Causes}

There are two distinct causes of social isolation among forced migrants. One cause is the physical displacement from one's home, which leads to alienation from one's family and other former social circles. Another cause, however, is cultural alienation in one's new environment \cite{northcote_breaking_2006}. Thus, forced migrants face isolation from two different angles at the same time, making it that much harder to avoid the negative effects of isolation.

Living in dedicated forced-migrant housing has been shown to cause shame among newcomer children and thus lead to their social isolation when they don't invite friends over to play. Members of established migrant communities sometimes exclude newcomers. Invitations into others' homes, common interests like sports, and mutual support are key to making social contact. Open classrooms and opportunities for expression are helpful, but not if the ``refugee-friendliness'' feels too fabricated \cite{anderson_you_2001}.


\subsection*{Patterns of Occurrence}

Among forced migrants, social isolation is prevalent in the early stages of resettlement as a consequence of not having meaningful and supportive relationships \cite{simich_social_2003}. It remains a common issue even after years of living in the host country, especially when other challenges of resettlement are not overcome. For example, those who are unable to learn the local language and those who have difficulty finding work often experience social isolation that only intensifies with time \cite[p.~166]{almohamed_designing_2016}. The problem becomes a vicious cycle because creating new relationships, learning the local language, and finding employment are all much harder when one is socially isolated.

While \citeA{northcote_breaking_2006} describe how social isolation can become a self-reinforcing negative cycle, in this thesis we explore if the opposite is also true. Following the same principles, social contact should also be self-reinforcing: contact with well-connected individuals could lead to exponentially more contact thereafter.

Some newcomers face greater risk of social isolation because of intersectional issues. Those with cultural traditions of gender separation have fewer opportunities to create new social contacts \cite{almohamed_designing_2016}. Women newcomers are more often socially isolated than men \cite{hynie_immigrant_2011}. Muslim refugee women are even further marginalized in Australian society due to animosity towards the Muslim expression of religious traditions \cite{casimiro_isolation_2007}.
% TODO: Find research about the elderly too?


\subsection*{Strategies for Addressing Social Isolation}

Some of the highest-quality social contact comes from co-ethnic peer support, and there is room to improve systems that support this kind of contact, especially between asylum seekers and refugees \cite{almohamed_vulnerability_2016}. However, in many cases newcomer peer support groups are quickly overwhelmed so inter-community contact is also valuable, and necessary. Many newcomers face social exclusion based on discrimination and a deficit of social resources \cite{hynie_immigrant_2011}. This makes contact with the part of society that is both responsible for the discrimination and in control of the majority of social resources all the more important.

Social isolation doesn't just come from an inability to reach out to locals. Newcomers also express a need for others to reach out to them first \cite[p.~166]{almohamed_designing_2016}. This means systems aimed at reducing social isolation should include communication channels in both directions.

New technologies are giving forced migrants the ability to reduce their social isolation. On the one hand, modern communication technologies help them maintain ties to family back home, which alleviates the first of the two aforementioned causes of social isolation. This is a dramatic change from the near total isolation that forced migrants of the previous generation experienced \cite[p.~553]{harney_precarity_2013}. In this paper, however, we are more interested in how technology can address the second cause of social isolation: disconnection from the host society. ICT has been shown to increase newcomers' ability to participate in the host culture's information society, communicate effectively in the host society, understand the new society, be socially connected and express cultural identity \cite{diaz_andrade_information_2016}. All of these factors increase newcomers' contact with society and thereby reduce social isolation.


\subsection*{Relation to Geographic Isolation}

Social isolation is not the same as geographic isolation, but evidence suggests that the two are linked. In particular, geographic isolation seems to be a contributing factor to social isolation. For example, geographic dispersion (i.e., intentional separation of newcomers from other forced migrants) is a governmental policy in some EU nations that is intended to speed up integration but in reality can increase mental health issues and inhibit the resettlement process when those who don't integrate quickly feel left behind. The geographic separation of couples often leads to relationship problems, which in turn commonly result in the loss of one's closest social contact: one's partner \cite[p.~939]{carballo_migration_1998}. As discussed previously, geographic isolation in dedicated forced-migrant housing can also intensify social isolation. The lack of a secure or ``normal'' residence alienates newcomers and inhibits social strategies like inviting others to your home.

Perhaps it is more accurate to call this ``geographic loneliness,'' since the newcomer feels spatially isolated despite being quite close to many other individuals. Still, it is worth noting the importance of  spatial factors when addressing social isolation.



% Most of the research on social isolation focuses on senior populations, since it is such a common problem among the elderly. This research is often applicable to the forced migrant experience as well. While forced migrant isolation may have different root causes, the contributing factors are comparable. Newcomers tend to have limited language abilities instead of limited physical abilities, mental health issues from trauma instead of from aging, loss of friends and family because of involuntary relocation instead of old age. It was no surprise when \citeA{beacker_technology_2014}'s finding that communication technology plays a critical role in reducing social isolation in the context of senior citizens matched the findings of the interviews in this study in the context of forced migration.

\section{ICT for Forced Migrants}
\label{sec:ict}


\subsection*{Value}

We already discussed how ICT can significantly reduce the first type of social isolation that forced migrants experience: the isolation from their country of origin \cite{harney_precarity_2013}. However ICT has also been shown to help newcomers build social capital in their new city, which is key when working to avoid the second type of social isolation that is often experienced after arrival \cite{alam_digital_2015}.

ICT has further been found to build trust by facilitating connections between newcomers and members of the host community, as well as by creating a sense of belonging in a group \cite{almohamed_vulnerability_2016}.


\subsection*{Challenges}

The design of ICT for newcomers involves the unique challenge of balancing multiple cultural contexts in one system \cite{almohamed_designing_2016}. Designers must also consider users who have a limited abilities in the local language and perhaps with literacy in general, limited internet access, a need to understand particularly complex compliance and geospatial information, a high need for reliability and timeliness, and limited experience with geospatial technologies \cite{bustamante_duarte_exploring_2018}.


\subsection*{Best Practices}

The following services were all developed to study different ways that ICT can ease the resettlement of forced migrants. In each case, the researchers identified some useful best practices when designing tech for newcomers.

\textit{Lantern} \cite{baranoff_lantern:_2015} is a service that allows forced migrants in the United States to learn about their new environment by scanning strategically placed NFC stickers with their phone. The creators highlight the value of having a system that is flexible enough to meet diverse and changing needs, is based on technology that forced migrants already have and use, leverages the expertise of more settled forced migrants to support newcomers, empowers newcomers to help themselves and thus reduces the burden on official support workers, and strengthens the overall feeling of community by giving newcomers the opportunity to ``give back'' to the system.

\textit{Rivrtran} \cite{brown_designing_2016} is a mobile app, also from the United States, that helps forced migrants to communicate with people who don't share a common language by providing access to volunteer interpreters. The authors found the strengths of such a service were that it facilitated communication between refugees and people who could provide support, simultaneously supporting and encouraging the users to eventually get by without such a service, and made use of a voice-based user interface to accommodate users with limited literacy. Of particular interest to our research on social isolation, they also found a major benefit of the app was that it supported forced migrants to build social capital simply by engaging in everyday conversation.

\textit{Tarjimly} is a similar service developed in the United States by Atif Javed, Aziz Alghunaim, and Abubakar Abid \cite{utley_how_2017}. It is a Facebook Messenger bot that connects refugees needing translation services with volunteers able to help in the moment. The service shows what happens when you give people a way to support forced migrants without being majorly inconvenienced: thousands of volunteers offer interpretation services that would normally cost a fortune. The creators chose to use Facebook Messenger as their platform because it is a tool that is already familiar to so many users.

\textit{Integreat} \cite{schreieck_supporting_2017} is a service designed to facilitate information dissemination to forced migrants in Germany. The authors outline a number of best practices when designing software for maximum information transmission in an intercultural context. While the focus of this paper is not on information transmission, Integreat's design principles are also useful when applied to user interfaces intended for other purposes. For example, icons should always be accompanied by text and services should be usable offline.

\textit{Moin} \cite{verbert_refugees_2016} is a gamified informal learning app, also from Germany. The authors highlight the value of using a human-centered design process when developing technology for use by forced migrants, and stress the importance of accommodating low language abilities. They also point out how usability depends not only on the design of the system but also the context of the user, and how the contextual differences of forced migrants and German locals meant the app was much less usable for forced migrants.


\citeA{bustamante_duarte_exploring_2018} worked with forced migrants and social workers in Münster, Germany, reviewing 36 apps and services and identifying several best practices and gaps. They observed a lack of services focused on non-Arabic-speakers and supporting offline use. They highlighted the importance of supporting multi-directional exchanges, user collaboration, flexible visualizations, and geovisualizations. They also recommended the use of open-source data platforms to create systems where knowledge can be built up over time by numerous contributors. Finally, they suggested the potential value of better geospatial technologies and location-based features, which up until now have been largely absent from ICT designed for forced migrants. The next section discusses this potential in greater depth.

\section{Geospatial Technologies for Forced Migrants}
\label{sec:geospatial_technologies}

\subsection*{Value}

Geospatial technologies can be of great value to forced migrants. One participant in our needs assessment study recounted:

\begin{displayquote}
When I fled from my homeland to Germany, my best friend was my cellphone. It really helped me and at the same time I helped many people because of my cellphone and my apps. I installed a navigator that worked offline on my phone and with maps from all the countries [that we were passing through]. I guided more than 100 people from various countries and the people trusted me because I already said [that] this [would be] happening like this and we are going to a place that is like this and when we got there [it was so].
\end{displayquote} % 00:25:30

In addition to supporting navigation during migration, mobile mapping services also help newcomers get around the cities where they resettle. % TODO: citation needed

Location-based social networks facilitate arranging to meet in person and the exploration of one’s environment \cite{lee_location-based_2013}. Location-based context filtering makes services easier to use by reducing information overload and tailoring what is offered to the individual user. This may be particularly powerful when the goal is to create social contact, because the probability of having a social connection between two individuals decreases with the distance between them \cite{scellato_socio-spatial_2011}.

\textit{Lantern}, the location-based information service described in the previous section, is an example of an LBS that does not make use of geospatial technologies. The location of the user is determined by knowing the location of the NFC sticker that they scan.

% TODO
% \subsection*{Challenges}

% \begin{itemize}
% \item Privacy
% \item Unfamiliarity with geospatial technologies
% \end{itemize}


% \subsection*{Best Practices}

% Build for users with limited experience with geospatial technologies.
\section{Freecycling}
\label{sec:freecycling}

\subsection*{Introduction}

In 2003, Deron Beal coined the term ``freecycle,'' a blend between the words ``free'' and ``recycle,'' when he founded The Freecycle Network in Tucson, Arizona, in the United States\footnote{https://www.freecycle.org/about/background}. Making things available for free via the internet, however, has been popular for much longer (think software, music, etc.) \cite{eden_blurring_2017}. There are now innumerable freecycling platforms all over the world. The Freecycle Network alone claims to have millions of members in more than 110 countries.

In Münster, the most notable freecycling platforms take the form of Facebook groups. The biggest is ``Verschenk's Münster,'' a giving-only platform with roughly 28,000 members as of this writing\footnote{\url{https://www.facebook.com/groups/473505729373257}}. The second largest is ``Foodsharing Münster,'' a group focused specifically on avoiding food waste, which currently had just over 8,000 members at the time of this writing\footnote{\url{https://www.facebook.com/groups/607791439294335}}. eBay Kleinanzeigen, a German subsidiary of the international eBay corporation, is a third popular platform for the free peer-to-peer exchange of goods\footnote{\url{https://themen.ebay-kleinanzeigen.de/ueber-uns/}}.

\subsection*{Value}

It has been shown that freecycling leads to trust-filled interactions between individuals with little in common other than their participation in a freecycling group \cite{nelson_trash_2009}. Freecycling also encourages civil engagement and members tend to be more politically engaged in society than the average citizen \cite{nelson_downshifting_2007}. Freecycling has also been shown to blur binary boundaries that are normally quite stark in such systems of exchange. Freecycling platforms bring together consumers and producers, givers and receivers, those who have resources and those who do not. Members can embody both roles at the same time. The platforms connect digital and material worlds, demonstrating the potential of virtual communities to have a real-world impact on people's lives. Finally, they unite mainstream and alternative cultures in one community \cite{eden_blurring_2017}. Therefore, participation in such communities could be of great value to socially isolated forced migrants, because these are some of the divisions that lead to their social isolation.

\subsection*{Challenges}

Current freecycling systems have several problems that should be addressed in a freecycling system intended to reduce social isolation. Organizational policies tend to tightly control the nature of exchanges due to an overemphasized focus on the environment. These policies contribute to the dominance of user communities by those seeking only ``green-washed convenience.'' Such user communities exhibit less altruism and solidarity when compared with other online groups \cite{aptekar_gifts_2016}.

Freecycling communities are also not the altruistic ``gift economies'' that platform administrators often claim them to be, since most members expect some kind of generalized reciprocity, i.e. a reciprocal reward that may only come after a delay or from a different person than the recipient of the original ``gift''. Transactions that take place through such systems are a ``hybridized form of exchange'', having characteristics of both gift giving and trading \cite{arsel_hybrid_2011}.

In order to create a real sense of solidarity (not charity) in generalized exchange platforms, the forging of a group identity is key \cite{willer_structure_2012}. Some groups forge such an identity through a common interest in sustainability, but that identity excludes people who could benefit from the service in practical ways, such as by acquiring things for free that they otherwise cannot afford \cite{aptekar_gifts_2016}.

\section{Human-Centered Design}
\label{sec:hcd}

The methodology of this work is based on the human-centered design process, as defined by the International Organization for Standardization  (\textit{ISO 9241-210:2010, Ergonomics of human-system interaction}). The principles of human-centered design have particular relevance when designing for vulnerable populations such as forced migrants. Language barriers, cultural differences, unfamiliarity with a new city – the same challenges which can lead to social isolation – can also keep forced migrants from having any input on the services that are designed for them. If the resulting systems are therefore unusable or of little practical value, forced migrants have one more challenge added to their plates.

Human-centered design leverages the knowledge of the true experts of a system – its users – to develop services of a higher quality. It focuses on improving ease of use and usefulness based on user evaluations of the system \cite{roth_user-centered_2015}. While \citeA{bustamante_duarte_participatory_2018} show that participatory design is even more empowering for users and produces software of an even higher quality, the organization and implementation of design workshops can be quite time intensive and did not fit into the scope of this six-month master's thesis. Thus, human-centered design was used with the intention of incorporating the maximum amount of user input within the time available.

% Section two could conclude with a summary subsection that interrelates what was discussed in chapter two and links this to the overall goal of your thesis.
\section{Summary}

In this chapter, we have attempted to summarize past research on social isolation in the context of forced migration and the use of ICT to address this and other forced migrant challenges. We highlighted areas that need more research, including 1) the development of services that are explicitly designed to reduce the social isolation of forced migrants and 2) the adaptation of existing promising technologies – such as geospatial technologies, location-based services, and freecycling systems – to the unique needs of forced migrants. Our review of this literature led us to our research question and guided our methodology, as explained in the chapter that follows.