\section{Freecycling}
\label{sec:freecycling}

\subsection*{Introduction}

In 2003, Deron Beal coined the term ``freecycle,'' a blend between the words ``free'' and ``recycle,'' when he founded The Freecycle Network in Tucson, Arizona, in the United States\footnote{https://www.freecycle.org/about/background}. Making things available for free via the internet, however, has been popular for much longer (think software, music, etc.) \cite{eden_blurring_2017}. There are now innumerable freecycling platforms all over the world. The Freecycle Network alone claims to have millions of members in more than 110 countries.

In Münster, the most notable freecycling platforms take the form of Facebook groups. The biggest is ``Verschenk's Münster,'' a giving-only platform with roughly 28,000 members as of this writing\footnote{\url{https://www.facebook.com/groups/473505729373257}}. The second largest is ``Foodsharing Münster,'' a group focused specifically on avoiding food waste, which currently had just over 8,000 members at the time of this writing\footnote{\url{https://www.facebook.com/groups/607791439294335}}. eBay Kleinanzeigen, a German subsidiary of the international eBay corporation, is a third popular platform for the free peer-to-peer exchange of goods\footnote{\url{https://themen.ebay-kleinanzeigen.de/ueber-uns/}}.

\subsection*{Value}

It has been shown that freecycling leads to trust-filled interactions between individuals with little in common other than their participation in a freecycling group \cite{nelson_trash_2009}. Freecycling also encourages civil engagement and members tend to be more politically engaged in society than the average citizen \cite{nelson_downshifting_2007}. Freecycling has also been shown to blur binary boundaries that are normally quite stark in such systems of exchange. Freecycling platforms bring together consumers and producers, givers and receivers, those who have resources and those who do not. Members can embody both roles at the same time. The platforms connect digital and material worlds, demonstrating the potential of virtual communities to have a real-world impact on people's lives. Finally, they unite mainstream and alternative cultures in one community \cite{eden_blurring_2017}. Therefore, participation in such communities could be of great value to socially isolated forced migrants, because these are some of the divisions that lead to their social isolation.

\subsection*{Challenges}

Current freecycling systems have several problems that should be addressed in a freecycling system intended to reduce social isolation. Organizational policies tend to tightly control the nature of exchanges due to an overemphasized focus on the environment. These policies contribute to the dominance of user communities by those seeking only ``green-washed convenience.'' Such user communities exhibit less altruism and solidarity when compared with other online groups \cite{aptekar_gifts_2016}.

Freecycling communities are also not the altruistic ``gift economies'' that platform administrators often claim them to be, since most members expect some kind of generalized reciprocity, i.e. a reciprocal reward that may only come after a delay or from a different person than the recipient of the original ``gift''. Transactions that take place through such systems are a ``hybridized form of exchange'', having characteristics of both gift giving and trading \cite{arsel_hybrid_2011}.

In order to create a real sense of solidarity (not charity) in generalized exchange platforms, the forging of a group identity is key \cite{willer_structure_2012}. Some groups forge such an identity through a common interest in sustainability, but that identity excludes people who could benefit from the service in practical ways, such as by acquiring things for free that they otherwise cannot afford \cite{aptekar_gifts_2016}.
