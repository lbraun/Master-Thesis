\section{Geospatial Technologies for Forced Migrants}
\label{sec:geospatial_technologies}

\subsection*{Value}

Geospatial technologies can be of great value to forced migrants. One participant in our needs assessment study recounted:

\begin{displayquote}
When I fled from my homeland to Germany, my best friend was my cellphone. It really helped me and at the same time I helped many people because of my cellphone and my apps. I installed a navigator that worked offline on my phone and with maps from all the countries [that we were passing through]. I guided more than 100 people from various countries and the people trusted me because I already said [that] this [would be] happening like this and we are going to a place that is like this and when we got there [it was so].
\end{displayquote} % 00:25:30

In addition to supporting navigation during migration, mobile mapping services also help newcomers get around the cities where they resettle. % TODO: citation needed

Location-based social networks facilitate arranging to meet in person and the exploration of one’s environment \cite{lee_location-based_2013}. Location-based context filtering makes services easier to use by reducing information overload and tailoring what is offered to the individual user. This may be particularly powerful when the goal is to create social contact, because the probability of having a social connection between two individuals decreases with the distance between them \cite{scellato_socio-spatial_2011}.

\textit{Lantern}, the location-based information service described in the previous section, is an example of an LBS that does not make use of geospatial technologies. The location of the user is determined by knowing the location of the NFC sticker that they scan.

% TODO
% \subsection*{Challenges}

% \begin{itemize}
% \item Privacy
% \item Unfamiliarity with geospatial technologies
% \end{itemize}


% \subsection*{Best Practices}

% Build for users with limited experience with geospatial technologies.