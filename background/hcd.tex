\section{Human-Centered Design}
\label{sec:hcd}

The methodology of this work is based on the human-centered design process, as defined by the International Organization for Standardization  (\textit{ISO 9241-210:2010, Ergonomics of human-system interaction}). The principles of human-centered design have particular relevance when designing for vulnerable populations such as forced migrants. Language barriers, cultural differences, unfamiliarity with a new city – the same challenges which can lead to social isolation – can also keep forced migrants from having any input on the services that are designed for them. If the resulting systems are therefore unusable or of little practical value, forced migrants have one more challenge added to their plates.

Human-centered design leverages the knowledge of the true experts of a system – its users – to develop services of a higher quality. It focuses on improving ease of use and usefulness based on user evaluations of the system \cite{roth_user-centered_2015}. While \citeA{bustamante_duarte_participatory_2018} show that participatory design is even more empowering for users and produces software of an even higher quality, the organization and implementation of design workshops can be quite time intensive and did not fit into the scope of this six-month master's thesis. Thus, human-centered design was used with the intention of incorporating the maximum amount of user input within the time available.