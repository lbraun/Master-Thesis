\section{ICT for Forced Migrants}
\label{sec:ict}


\subsection*{Value}

We already discussed how ICT can significantly reduce the first type of social isolation that forced migrants experience: the isolation from their country of origin \cite{harney_precarity_2013}. However ICT has also been shown to help newcomers build social capital in their new city, which is key when working to avoid the second type of social isolation that is often experienced after arrival \cite{alam_digital_2015}.

ICT has further been found to build trust by facilitating connections between newcomers and members of the host community, as well as by creating a sense of belonging in a group \cite{almohamed_vulnerability_2016}.


\subsection*{Challenges}

The design of ICT for newcomers involves the unique challenge of balancing multiple cultural contexts in one system \cite{almohamed_designing_2016}. Designers must also consider users who have a limited abilities in the local language and perhaps with literacy in general, limited internet access, a need to understand particularly complex compliance and geospatial information, a high need for reliability and timeliness, and limited experience with geospatial technologies \cite{bustamante_duarte_exploring_2018}.


\subsection*{Best Practices}

The following services were all developed to study different ways that ICT can ease the resettlement of forced migrants. In each case, the researchers identified some useful best practices when designing tech for newcomers.

\textit{Lantern} \cite{baranoff_lantern:_2015} is a service that allows forced migrants in the United States to learn about their new environment by scanning strategically placed NFC stickers with their phone. The creators highlight the value of having a system that is flexible enough to meet diverse and changing needs, is based on technology that forced migrants already have and use, leverages the expertise of more settled forced migrants to support newcomers, empowers newcomers to help themselves and thus reduces the burden on official support workers, and strengthens the overall feeling of community by giving newcomers the opportunity to ``give back'' to the system.

\textit{Rivrtran} \cite{brown_designing_2016} is a mobile app, also from the United States, that helps forced migrants to communicate with people who don't share a common language by providing access to volunteer interpreters. The authors found the strengths of such a service were that it facilitated communication between refugees and people who could provide support, simultaneously supporting and encouraging the users to eventually get by without such a service, and made use of a voice-based user interface to accommodate users with limited literacy. Of particular interest to our research on social isolation, they also found a major benefit of the app was that it supported forced migrants to build social capital simply by engaging in everyday conversation.

\textit{Tarjimly} is a similar service developed in the United States by Atif Javed, Aziz Alghunaim, and Abubakar Abid \cite{utley_how_2017}. It is a Facebook Messenger bot that connects refugees needing translation services with volunteers able to help in the moment. The service shows what happens when you give people a way to support forced migrants without being majorly inconvenienced: thousands of volunteers offer interpretation services that would normally cost a fortune. The creators chose to use Facebook Messenger as their platform because it is a tool that is already familiar to so many users.

\textit{Integreat} \cite{schreieck_supporting_2017} is a service designed to facilitate information dissemination to forced migrants in Germany. The authors outline a number of best practices when designing software for maximum information transmission in an intercultural context. While the focus of this paper is not on information transmission, Integreat's design principles are also useful when applied to user interfaces intended for other purposes. For example, icons should always be accompanied by text and services should be usable offline.

\textit{Moin} \cite{verbert_refugees_2016} is a gamified informal learning app, also from Germany. The authors highlight the value of using a human-centered design process when developing technology for use by forced migrants, and stress the importance of accommodating low language abilities. They also point out how usability depends not only on the design of the system but also the context of the user, and how the contextual differences of forced migrants and German locals meant the app was much less usable for forced migrants.


\citeA{bustamante_duarte_exploring_2018} worked with forced migrants and social workers in Münster, Germany, reviewing 36 apps and services and identifying several best practices and gaps. They observed a lack of services focused on non-Arabic-speakers and supporting offline use. They highlighted the importance of supporting multi-directional exchanges, user collaboration, flexible visualizations, and geovisualizations. They also recommended the use of open-source data platforms to create systems where knowledge can be built up over time by numerous contributors. Finally, they suggested the potential value of better geospatial technologies and location-based features, which up until now have been largely absent from ICT designed for forced migrants. The next section discusses this potential in greater depth.
