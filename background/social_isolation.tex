\section{Social Isolation and Forced Migrants}
\label{sec:social_isolation}


\subsection*{Introduction}

Social isolation is defined as ``the absence of social interactions, contacts, and relationships with family and friends, with neighbors on an individual level, and with `society at large' on a broader level'' \cite{berg_social_1992}.

According to \citeA{beacker_technology_2014}, social isolation is not the same thing as loneliness. One can feel lonely even when one has frequent contact with others. One can also choose to isolate oneself from others and not feel lonely. However, when one is forced into social isolation, as is often the case with forced migrants, the isolation commonly leads to loneliness.

\citeA{cornwell_measuring_2009} consider loneliness to be a part of social isolation, but still divide social isolation into two components: social disconnectedness and perceived isolation. Social disconnectedness implies either a small social network or a lack of activity within one's social network. Perceived isolation comes from a lack of support and loneliness.

Similarly, forced migrant integration can be divided into two categories, practical integration and emotional integration \cite{colic-peisker_croatians_2002}. Social disconnectedness is an obstacle to practical integration and loneliness is an obstacle to emotional integration. Clearly, both are important and interrelated. In this thesis, however, we focus primarily on social disconnectedness because it is simpler to measure (e.g. by counting instances of social contact) and because changes are more readily observed, even over a short period of time.


\subsection*{Causes}

There are two distinct causes of social isolation among forced migrants. One cause is the physical displacement from one's home, which leads to alienation from one's family and other former social circles. Another cause, however, is cultural alienation in one's new environment \cite{northcote_breaking_2006}. Thus, forced migrants face isolation from two different angles at the same time, making it that much harder to avoid the negative effects of isolation.

Living in dedicated forced-migrant housing has been shown to cause shame among newcomer children and thus lead to their social isolation when they don't invite friends over to play. Members of established migrant communities sometimes exclude newcomers. Invitations into others' homes, common interests like sports, and mutual support are key to making social contact. Open classrooms and opportunities for expression are helpful, but not if the ``refugee-friendliness'' feels too fabricated \cite{anderson_you_2001}.


\subsection*{Patterns of Occurrence}

Among forced migrants, social isolation is prevalent in the early stages of resettlement as a consequence of not having meaningful and supportive relationships \cite{simich_social_2003}. It remains a common issue even after years of living in the host country, especially when other challenges of resettlement are not overcome. For example, those who are unable to learn the local language and those who have difficulty finding work often experience social isolation that only intensifies with time \cite[p.~166]{almohamed_designing_2016}. The problem becomes a vicious cycle because creating new relationships, learning the local language, and finding employment are all much harder when one is socially isolated.

While \citeA{northcote_breaking_2006} describe how social isolation can become a self-reinforcing negative cycle, in this thesis we explore if the opposite is also true. Following the same principles, social contact should also be self-reinforcing: contact with well-connected individuals could lead to exponentially more contact thereafter.

Some newcomers face greater risk of social isolation because of intersectional issues. Those with cultural traditions of gender separation have fewer opportunities to create new social contacts \cite{almohamed_designing_2016}. Women newcomers are more often socially isolated than men \cite{hynie_immigrant_2011}. Muslim refugee women are even further marginalized in Australian society due to animosity towards the Muslim expression of religious traditions \cite{casimiro_isolation_2007}.
% TODO: Find research about the elderly too?


\subsection*{Strategies for Addressing Social Isolation}

Some of the highest-quality social contact comes from co-ethnic peer support, and there is room to improve systems that support this kind of contact, especially between asylum seekers and refugees \cite{almohamed_vulnerability_2016}. However, in many cases newcomer peer support groups are quickly overwhelmed so inter-community contact is also valuable, and necessary. Many newcomers face social exclusion based on discrimination and a deficit of social resources \cite{hynie_immigrant_2011}. This makes contact with the part of society that is both responsible for the discrimination and in control of the majority of social resources all the more important.

Social isolation doesn't just come from an inability to reach out to locals. Newcomers also express a need for others to reach out to them first \cite[p.~166]{almohamed_designing_2016}. This means systems aimed at reducing social isolation should include communication channels in both directions.

New technologies are giving forced migrants the ability to reduce their social isolation. On the one hand, modern communication technologies help them maintain ties to family back home, which alleviates the first of the two aforementioned causes of social isolation. This is a dramatic change from the near total isolation that forced migrants of the previous generation experienced \cite[p.~553]{harney_precarity_2013}. In this paper, however, we are more interested in how technology can address the second cause of social isolation: disconnection from the host society. ICT has been shown to increase newcomers' ability to participate in the host culture's information society, communicate effectively in the host society, understand the new society, be socially connected and express cultural identity \cite{diaz_andrade_information_2016}. All of these factors increase newcomers' contact with society and thereby reduce social isolation.


\subsection*{Relation to Geographic Isolation}

Social isolation is not the same as geographic isolation, but evidence suggests that the two are linked. In particular, geographic isolation seems to be a contributing factor to social isolation. For example, geographic dispersion (i.e., intentional separation of newcomers from other forced migrants) is a governmental policy in some EU nations that is intended to speed up integration but in reality can increase mental health issues and inhibit the resettlement process when those who don't integrate quickly feel left behind. The geographic separation of couples often leads to relationship problems, which in turn commonly result in the loss of one's closest social contact: one's partner \cite[p.~939]{carballo_migration_1998}. As discussed previously, geographic isolation in dedicated forced-migrant housing can also intensify social isolation. The lack of a secure or ``normal'' residence alienates newcomers and inhibits social strategies like inviting others to your home.

Perhaps it is more accurate to call this ``geographic loneliness,'' since the newcomer feels spatially isolated despite being quite close to many other individuals. Still, it is worth noting the importance of  spatial factors when addressing social isolation.



% Most of the research on social isolation focuses on senior populations, since it is such a common problem among the elderly. This research is often applicable to the forced migrant experience as well. While forced migrant isolation may have different root causes, the contributing factors are comparable. Newcomers tend to have limited language abilities instead of limited physical abilities, mental health issues from trauma instead of from aging, loss of friends and family because of involuntary relocation instead of old age. It was no surprise when \citeA{beacker_technology_2014}'s finding that communication technology plays a critical role in reducing social isolation in the context of senior citizens matched the findings of the interviews in this study in the context of forced migration.
