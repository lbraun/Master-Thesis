\chapter{Needs Assessment Study}
\label{cha:needs}


\section{Introduction and Objective}

This section covers the first of the three phases of our work to answer our research question. The objective of this first phase was to assess the needs of forced migrants with regards to avoiding social isolation during resettlement.

This needs assessment was necessary because in order to find out how a location-based freecycling application can reduce the social isolation of forced migrants, it is important to identify the strategies forced migrants currently use to avoid social isolation, the associated challenges, and what is needed to overcome these challenges and strengthen the current catalysts of social contact.

A second objective of this study was to assess the needs of the users and moderators of existing freecycling platforms in Münster. An understanding of those needs is essential to the design of a useful and usable location-based freecycling service.

In terms of the human-centered design process, this phase covered steps 1, 2, and 3 (planning, gathering context, and developing requirements). We group them together into one chapter for the sake of simplicity, but still performed them separately in practice, noting the importance of all three steps: careful preparation and context analysis produce usage requirements that are traceable back to the users' reality \cite{riedemann_context_2018}.

After talking with experts, we decided to implement the human-centered design process with three user groups. We used in-person interviews to understand and specify the context of use for each group. Finally, we analyzed recordings of the interviews to extract implied needs and usage requirements. Following is an in-depth explanation of each step.


\section{Participants}

In preliminary discussions with experts on forced migrant resettlement and freecycling in Münster, we decided to focus on triangulating the needs of three user groups to develop a user-centered application. The first group was forced migrants living in the general vicinity of Münster. The second group was residents of Münster who participate in some capacity in one of the several freecycling communities in Münster. The third group was the moderators of said groups.

These user groups represented the three perspectives we needed to answer our research question, each group holding the expertise to answer a different critical sub-question. Forced migrants know how to make social contacts as a newcomer. Freecyclers know how to freecycle. And freecycling moderators know how to make a freecycling service safe and easy to use.

We recruited forced migrant participants by approaching acquaintances that we met through mutual friends and by attending newcomer-friendly meet-ups. We contacted the moderators of freecycling services through their publicly available online contact information and obtained their permission to recruit freecyclers by posting directly in the freecycling forums. Further participants joined the study through snowball sampling. For each user group, we continued interviewing until we achieved information saturation; that is, until participants' responses were no longer yielding significant new information about the focus topic.

Five of the interviewees were forced migrants, three were members of existing freecycling platforms in Münster, and three were moderators of these platforms.

Of the five forced migrant participants, two were between the ages of 18 and 25, and three between 26 and 35. Two migrated from Syria and the other three from Iran, Iraq, and Turkey. We interviewed four men and one woman.

The freecycling interviewees represented a broader age range, between 23 and 50 years old. They all came from Germany except one freecycler who immigrated to Münster from Eastern Europe for his studies. Four of the freecyclers were women and two were men.


\section{Procedure}

As explained in section~\ref{sec:hcd}, the first step of the human-centered design process is to understand and specify the context in which the system will be used. For this so-called ``context analysis,'' it is common practice to perform a kind of loosely structured interview with various members of the user community known as a ``context interview'' \cite{riedemann_context_2018}. For this study we performed a total of eleven context interviews over a period of 22 days in November 2018. The script we followed for each interview can be found in appendix~\ref{app:interview-script}.

The interviews followed the semi-structured focused interview method. This allowed us to maintain non-directive management of the conversation and still gather very specific information about the topic in question \cite{flick_companion_2004}. The objective of the interviews was to understand the \textit{context} in which the user community performs the \textit{focus activity}.

The focus activity was different for the different user groups. It should be noted that this is unusual for human-centered design context interviews. Normally, the interviews address several different user groups but always focus on the same activity \cite{riedemann_context_2018}. In this study, however, such a structure was not possible. None of the forced migrant participants in the study had experience with freecycling and none of the freecycling participants had experience with resettling in Münster after a forced migration. We explored the possibility of asking all groups about ``gift giving'' or about ``making social connections,'' but pilot interviews showed that these focus topics were too vague to elicit useful responses. Asking about more specific and relatable activities produced more detailed and consistent results.

The forced migrant interviews focused on making social contact as a newcomer in Münster. The freecycler interviews focused on freecycling and the moderator interviews focused on moderating freecycling. In the end, all user groups gave information about all three topics. Forced migrants and freecyclers talked about the important role of the moderators in their communities. Freecyclers and moderators talked about making friends through their freecycling systems. Moderators talked about their own freecycling practices and forced migrants described the importance of giving generously when making social contacts in a new place. Thus, we identified the context of use for a service that both enables freecycling and reduces social isolation, although no such service existed in Münster at the time.

Context of use is made up of three elements: user goals, user tasks, and user environment \cite{maguire_context_2001}. Therefore, in order to perform step two of the human-centered design process, one must answer three different questions:

\begin{enumerate}
    \item Goals: \textit{What specific goals is the user trying to achieve?}
    \item Tasks: \textit{What are the main tasks involved in achieving those goals?}
    \item Environment: \textit{What are the unique characteristics of the environment in which the tasks are performed?}
\end{enumerate}

The interviews in this study were made up of five questions, each with its own purpose and each helping to gather information about some of the three context elements (see table~\ref{tab:interview_questions}).

\begin{table}[ht]
\begin{tabular}{|l|p{.5\textwidth}|l|}
\hline
\rowcolor{lightgray}
\textbf{Question type} & \textbf{Purpose}                                                                  & \textbf{Context elements} \\ \hline
1. Opening                                       & Warm up, get to know the person, make them comfortable.                                                                & Environment                                         \\ \hline
2. Involvement                                   & Understand how the person currently relates to the focus activity, how they can be supported by a system, and how not. & Environment, goals                                  \\ \hline
3. Prerequisites                                 & Understand challenges and needs when performing the tasks involved.                                                    & Tasks, environment                                  \\ \hline
4. Execution                                     & Understand the tasks involved, accompanying challenges, and relevant safety concerns.                                  & Tasks, goals                                        \\ \hline
5. Closing                                       & Learn about problems with existing systems, probe users for solutions.                                                 & Environment, goals                                  \\ \hline
\end{tabular}
\caption{Context interview question types and their purposes}
\label{tab:interview_questions}
\end{table}

The main guiding questions were supplemented with probing questions in case the user needed more prompting to cover the topic. The full sheet of guiding questions for each user group can be found in appendix~\ref{app:interviews}.


\section{Analysis}

The results of the interviews served as the basis for the rest of the human-centered design process. Learning about the context of use allowed us to identify the core tasks that would need to be performed in the system we were designing. The core tasks, in turn, served as a foundation for all of the subsequent elements of the design process:

\begin{center}
{\small
    context $\rightarrow$
    core tasks $\rightarrow$
    implied needs $\rightarrow$
    usage requirements $\rightarrow$
    design
}
\end{center}

In order to process the interviews, we partially transcribed the recordings into what are known as \textit{context scenarios}. Context scenarios are summaries of the interview responses, formulated in an easy-to-understand and vivid way. Each interview yields one context scenario, so there is overlap between the scenarios. Repeat information is consolidated in the next step. Context scenarios serve as the building blocks from which we derive both implied needs and usage requirements \cite{riedemann_context_2018}.

As shown in table~\ref{tab:context_analysis_elements}, context scenarios are made up of narrative statements with identifying user information (such as names) removed. An example context scenario can be found in appendix~\ref{app:context_scenario}.

The next step is to break down the interviewee's narrative descriptions of their activities into their core tasks and other context elements. This coding method is similar to descriptive coding or process coding \cite{saldana_coding_2009}, but focuses specifically on ``tasks'' or actions taken by the interviewee. The core tasks are phrased as a short verb phrase in the present tense, such as ``plan spontaneous meet-ups,'' ``introduce yourself,'' or ``find out about open/safe social opportunities''. These were repeated quite often across interviews, allowing the final core task list to be consolidated quite a bit by removing duplicates.

\begin{table}[ht]
\centering
\begin{tabular}{|l|l|}
\hline
\rowcolor{lightgray}
\textbf{Context analysis element} & \textbf{Structure}                                     \\ \hline
Context scenario                  & User \textless{}verb in present tense\textgreater{}... \\ \hline
Core tasks                        & \textless{}Verb phrase in present tense\textgreater{}  \\ \hline
Implied needs                     & In order to ... , ... must ... .                       \\ \hline
Usage requirements                & The user must be able to ... the system ...            \\ \hline
\end{tabular}
\caption{Context analysis elements and their structure}
\label{tab:context_analysis_elements}
\end{table}

The core tasks translated naturally into implied needs. Implied needs were determined by analyzing each core task and identifying the conditions that must be met in order for the task to be completable. These conditions were pulled directly from the context scenarios, so they can be linked back to the words of the interviewee.

From implied needs to usage requirements is a small leap. The translation is mostly a grammatical shift to phrase the implied need in terms of a user and an unspecified system.


\section{Results}
\label{sec:needs-results}

\subsubsection*{Context}

We extracted a total of 114 context elements from the five interviews with forced migrants. The five most frequently occurring codes were:

\begin{itemize}
    \item Characteristic: limited language ability
    \item Task: Meet an ``introducer''
    \item Task: Find people with common interests
    \item Task: Plan spontaneous meetups
    \item Task: Invite others to your home
\end{itemize}

Tables with all codes and their occurrence rates can be found in appendix~\ref{app:context_elements}.

\subsubsection*{Implied Needs}

We identified 55 different implied needs for the forced migrant user group. The five most prominent needs, each expressed by all five forced migrant interviewees, were:

\begin{itemize}
    \item accommodation of low language abilities,
    \item contact with an ``introducer'',
    \item contact with people with common interests,
    \item possibilities for spontaneous social contact, and
    \item ability to invite others to their home.
\end{itemize}

A table with all of the needs we identified can be found in appendix~\ref{app:implied_needs}.

\subsubsection*{User Requirements}

Out of the context and needs we derived a long list of user requirements for a location-based service to reduce the social isolation of forced migrants, for example:

\begin{itemize}
    \item The user must be able to choose the language of the system.
    \item The user must be able to see who is an ``introducer'' in the system.
    \item The user must be able to tell the system that they are available to meet up spontaneously.
    \item The user must be able to see other users in the system that are available to meet up spontaneously.
\end{itemize}


The full list of user requirements can be found in appendix~\ref{app:user_requirements}.

\subsubsection*{Overlap between User Groups}

We performed the same needs assessment with the freecycler and moderator user groups as well. This was necessary in order to develop a service that met the needs of all users.

From the analysis of the freecycler and moderator context interviews we identified many essential user requirements for any freecycling system, for example:

\begin{itemize}
    \item The user must be able to upload a photo.
    \item The user must be able to report misuse.
    \item The user must be able to mark an offer as taken.
\end{itemize}

We also found significant overlap between the contexts of use of the different user groups. For example, all three user groups reported the following shared context elements:

\begin{itemize}
    \item Goal: increasing personal happiness
    \item Risk: uncertainty about strangers
    \item Philosophy: patience and openness
    \item Task: invite others into your home
    \item Tool: smartphones
\end{itemize}

This shared context has important implications for location-based freecycling services aimed at reducing the social isolation of forced migrants.
