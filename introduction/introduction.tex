\chapter{Introduction}
\label{cha:introduction}

Forced migrants (including refugees, internally displaced people and asylum seekers) face enormous challenges when they arrive in the city where they settle. The geography, language, and the other residents of their new home are often completely unknown. This upheaval has numerous negative consequences, but one major theme is social isolation \cite{almohamed_designing_2016}. Social isolation is closely linked with poor psychological health and is thus an obstacle to forced migrant well-being during resettlement \cite{simich_social_2003, schweitzer_trauma_2006}.

Several studies have investigated how forced migrant resettlement can be eased by the use of novel technologies \cite{almohamed_vulnerability_2016, schreieck_supporting_2017, verbert_refugees_2016, brown_designing_2016}. \citeA{alam_digital_2015} found that digital inclusion and social inclusion are linked issues for refugees. A study in New Zealand found that in order to promote the social inclusion of refugees, researchers and policy makers \textit{must} consider information and communication technologies (ICT) \cite{diaz_andrade_information_2016}. On the other hand, forced migrants face many unique challenges when using novel technologies, for example, making do with limited access to internet, navigating text-rich services despite limited functional literacy, and overcoming their limited experience with geospatial services \cite{bustamante_duarte_exploring_2018}. To date, little to no research has focused on the appropriate design and implementation of geospatial services for this vulnerable user group.

One unexplored type of geospatial service, the location-based service (LBS), has the potential to address the social isolation of forced migrants in several ways. Location-based social networks facilitate arranging to meet in person and the exploration of one’s environment \cite{lee_location-based_2013}. Location-based context filtering makes services easier to use by reducing information overload and tailoring what is offered to the individual user. Tailoring to a local context may be particularly powerful when the goal is to create social contact, because the probability of social connection between two individuals decreases with the physical distance between them \cite{scellato_socio-spatial_2011}. Even in online social networks, the physical location of a person's social contacts can be used to accurately estimate the person's own location  \cite{backstrom_find_2010}.

In this thesis, we argue that freecycling platforms are one kind of location-based service with unique potential to reduce the social isolation of forced migrants. Freecycling is the act of getting rid of something you do not need any more by giving it to someone else who does need it, with no financial charge. Freecycling platforms depend on knowing users' locations in order to provide users with listings that can be picked up without undue cost to the collector. Such co-located exchanges of material goods, both among forced migrants and between forced migrants and locals who are not forced migrants, could address social isolation quite effectively.

In addition to its environmental benefits and appeal to people with limited resources, freecycling leads to trust-filled interactions outside of kin groups \cite{nelson_trash_2009}, encourages civil engagement \cite{nelson_downshifting_2007}, and blurs binary boundaries of needs and consumption \cite{eden_blurring_2017}. All of these benefits have significant potential to support forced migrant resettlement, but have yet to be researched as such.

Thus, the research question of this thesis is: \textit{How can a location-based freecycling application reduce the social isolation of forced migrants?}

The objective of the thesis was threefold: first, to identify and triangulate the needs of forced migrants, non-migrant locals, and moderators with respect to a freecycling service; second, to develop a location-based freecycling service that addresses those needs; and third, to evaluate the usability of the service and develop design recommendations for future location-based services that have forced migrants as an intended audience.

% \section{Study Description}

The study was broken up into three phases: needs identification, prototype development, and prototype evaluation. Throughout the whole study, we were guided by the human-centered design process defined by the International Organization for Standardization in \citeA{noauthor_iso_2010}.

We used interviews to identify the unique and intersecting needs of three user groups: forced migrants, local freecyclers, and freecycling moderators. We developed user requirements for a theoretical system from those needs. We found a great deal of overlap between the forced migrant and the freecycler context in Münster, Germany. This allowed us to design a system that simultaneously attempted to address both parties' needs.

We built a prototype mobile app based on a subset of the user requirements. We used free and open source technologies to show that development of such a service is feasible without great financial investment. We came up with a number of design recommendations for LBS for forced migrants during this process.

We evaluated the prototype during a two-week trial involving 6 forced migrants and 16 other residents of Münster. After the trial, we surveyed all participants about the usability of the service. We also surveyed participants and analyzed data logged on their devices to assess the potential of the service to reduce social isolation.

We found that the service was quite usable by both user groups. The prototype showed potential for reducing the social isolation of forced migrants in Münster, however a longer evaluation period with more participants is necessary to give conclusive results in that regard.

The remainder of this thesis is organized into seven more chapters. The next chapter, chapter~\ref{cha:background}, provides an overview of related work and background information on social isolation, technology for forced migrants, freecycling, and human-centered design. Chapter~\ref{cha:research_method} explains our overall research method and gives an overview of the three phases of the study, which are then explained in detail in the following three chapters. Chapter~\ref{cha:needs} describes how we assessed and analyzed the needs of forced migrants and freecyclers in Münster, and the results of that process. Chapter~\ref{cha:prototype} describes the implementation of a prototype location-based freecycling service using the findings of the needs assessment. Chapter~\ref{cha:evaluation} reports on the evaluation of the service. Chapter~\ref{cha:discussion} highlights the most significant findings of this work and discusses the broader implications of these results in the context of the field of geospatial technologies. Finally, chapter~\ref{cha:conclusion} summarizes the thesis with suggestions for future research on the topic.
% TODO: is this description of the last two chapters really accurate?