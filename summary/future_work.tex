\section{Future Work}
\label{sec:future_work}

There is great opportunity to continue the work started in the research conducted for this thesis.

Many of the user requirements developed in phase one of this work remain unimplemented in the prototype we built, and these could be added to the Geofreebie app. Further needs and user requirements came to light during the evaluation phase and these could also be addressed through additional development and implementation. This kind of iteration in the human-centered design process is expected and would produce higher-quality design solutions.

The prototype would benefit from further evaluation on a larger scale. Not just with more participants, but also with more time and perhaps with trials in other cities, the degree of usefulness of the app would become clearer.

Our analysis highlights the need for more research on technology that addresses the social isolation of forced migrants, especially those who are the most isolated. We only reached forced migrants who appeared to already have strong social networks. While their insights on the development of these networks was key to our research, those who have not succeeded in overcoming social isolation will be deprived even further of opportunities if their needs are not also considered in such services.