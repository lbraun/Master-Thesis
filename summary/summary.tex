\chapter{Conclusion}
\label{cha:conclusion}

\section{Contributions}

This work contributes to the greater understanding of challenges facing forced migrants today. We identified the context, core tasks, and specific needs of this vulnerable yet creative group in regards to creating social contacts in their new city and thus reducing their risk of social isolation.

We developed design solutions for a location-based service that supports forced migrant resettlement by helping them build their social network. We implemented some of these solutions and published them as open source software to allow adaptation and continued development.

Through the design and implementation of this freecycling app, we developed novel solutions to fill the gaps left by current freecycling services. These included independence from proprietary platforms, use of geospatial and location-based features, the ability to give feedback about other users, and recognition of inclusion and community-building as key goals when working towards the more commonly stated freecycling goal of sustainability.

Finally, we hope to have contributed insights that are useful for future development of location-based freecycling services for forced migrants, based on our prototype evaluation. We have demonstrated how to design an LBS that is considered usable by both forced migrants and German locals when they use it in their everyday lives. We have shown the potential of such a service to reduce social isolation in a real world scenario.

\section{Lessons Learned}

The first lesson we would like to share with future researchers is that working with forced migrants is not just the best way to address their needs, but also rewarding in its own right. The forced migrants we worked with were generous with their time and ideas. They were aware of the challenges and needs of newcomers and patiently kind as they explained these in the interviews.

Another important lesson was that recruiting participants from a Facebook group can be challenging and unfruitful. For example, we posted one request for interviews in the largest and most active freecycling group, with more than 27,900 members and on average 17 posts per day. Only 10 members responded to this call and all but 1 of those backed out before meeting up to talk.

Finally, we learned that our choice of technologies for the prototype app created development friction when combined. Based on the relatively small amount of documentation and support for native Cordova apps implemented in React, we believe this is one of the less common web framework choices for Cordova developers. Leaflet and Auth0 provide powerful libraries for vanilla JavaScript, but not for React. The adaptations that we implemented came with limitations and the general technical support one finds in question-and-answer forums was lacking, probably due to the lack of other developers trying to do the same thing.

\section{Future Work}
\label{sec:future_work}

There is great opportunity to continue the work started in the research conducted for this thesis.

Many of the user requirements developed in phase one of this work remain unimplemented in the prototype we built, and these could be added to the Geofreebie app. Further needs and user requirements came to light during the evaluation phase and these could also be addressed through additional development and implementation. This kind of iteration in the human-centered design process is expected and would produce higher-quality design solutions.

The prototype would benefit from further evaluation on a larger scale. Not just with more participants, but also with more time and perhaps with trials in other cities, the degree of usefulness of the app would become clearer.

Our analysis highlights the need for more research on technology that addresses the social isolation of forced migrants, especially those who are the most isolated. We only reached forced migrants who appeared to already have strong social networks. While their insights on the development of these networks was key to our research, those who have not succeeded in overcoming social isolation will be deprived even further of opportunities if their needs are not also considered in such services.
